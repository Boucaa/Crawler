\documentclass[a4]{article}
\usepackage[czech]{babel}
\usepackage[utf8]{inputenc}
\usepackage{graphicx}
\usepackage{a4wide}
\usepackage{amsmath,amsfonts,amssymb,amsthm}
%TODO pridat geometry a udelat spravne okraje
\begin{document}

%titulni strana
\begin{titlepage}
\begin{center}
{\Huge\textsc{Gymnázium Jana Keplera}\\}
{\large{obor vzdělání Gymnázium 79-41-K/81}\\[0.7cm]}
{\huge{Maturitní práce z informatiky}\\[0.5cm]}
{\Large{Jan Bouček}\\}
{\large{vedoucí práce: Tomáš Obdržálek}\\}
{\large{Praha 2017}}
\end{center}
\end{titlepage}

%prohlaseni
\newpage
Prohlašuji, že jsem jediným autorem této maturitní práce a všechny citace, použitá literatura a další zdroje jsou v práci uvedené. Tímto dle zákona 121/2000 Sb. (tzv. Autorský zákon) ve znění pozdějších předpisů uděluji bezúplatně škole Gymnázium Jana Keplera, Praha 6, Parléřova 2 oprávnění k výkonu práva na rozmnožování díla (§~13) a práva na sdělování díla veřejnosti (§~18) na dobu časově neomezenou a bez omezení územního rozsahu.\\[0.7cm]
\vspace{10cm}
{\large{V Praze dne 24. 3. 2017} \hfill Jan Bouček}
%anotace
\newpage
%TODO zrušit mezeru na začátku
{\Large\textbf{Anotace}\par}
Jeden z problémů moderní robotiky je chůze, čehož se tato práce narozdíl od konvenčních metod snaží dosáhnout pomocí strojového učení, konkrétně neuroevolučního algoritmu s nepřímým kódováním. Ten za pomoci techniky HyperNEAT generuje simulaci primitivního mozku, který chůzi robota ovládá.\par

{\Huge{POŘEŠIT, PŘELOŽIT}}
%samotna prace
\title{Vývoj chůze pomocí neuroevolučních algoritmů}
\author{Jan Bouček}
\date{}
\maketitle

\section{Úvod}

Evoluční algoritmy se požívají na problémy, jejichž exaktní řešení neznáme, nebo je reálě nemožné. Jsou až tak blízké procesům v přírodě, že je často můžeme využít a k znovunalezení řešení, která v přírodě už existují. Právě proto se evoluční technika hodí pro vývoj chůze.

\section{Teorie}
\section{Výsledky}
\section{Implementace?}%sem?
\section{Závěr}
\section{Uživatelská dokumentace}
\section{Diskuze?}
\end{document}