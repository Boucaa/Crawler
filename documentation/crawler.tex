\documentclass[a4]{article} 
\usepackage[czech]{babel} 
\usepackage[utf8]{inputenc} 
\usepackage{graphicx} 
\usepackage{a4wide} 
\usepackage{amsmath,amsfonts,amssymb,amsthm} 
%TODO pridat geometry a udelat spravne okraje 
\begin{document} 
 
%titulni strana 
\begin{titlepage} 
\begin{center} 
{\huge\textsc{Gymnázium Jana Keplera}\\} 
{\large{obor vzdělání Gymnázium 79-41-K/81}\\[0.7cm]} 
{\Huge{Maturitní práce z informatiky}\\[0.5cm]} 
{\Large{Jan Bouček}\\} 
{\large{vedoucí práce: Tomáš Obdržálek}\\} 
{\large{Praha 2017}} 
\end{center} 
\end{titlepage} 
 
%prohlaseni 
\newpage 
Prohlašuji, že jsem jediným autorem této maturitní práce a všechny citace, použitá literatura a další zdroje jsou v práci uvedené. Tímto dle zákona 121/2000 Sb. (tzv. Autorský zákon) ve znění pozdějších předpisů uděluji bezúplatně škole Gymnázium Jana Keplera, Praha 6, Parléřova 2 oprávnění k výkonu práva na rozmnožování díla (§~13) a práva na sdělování díla veřejnosti (§~18) na dobu časově neomezenou a bez omezení územního rozsahu.\\[0.7cm] 
\vspace{10cm} 
{\large{V Praze dne 24. 3. 2017} \hfill Jan Bouček} 
%anotace 
\newpage 
\tableofcontents
\newpage
%TODO zrušit mezeru na začátku 
{\Large\textbf{Anotace}\par}
Jeden z problémů moderní robotiky je chůze, čehož se tato práce narozdíl od konvenčních metod snaží dosáhnout pomocí strojového učení, konkrétně neuroevolučního algoritmu s nepřímým kódováním. Ten za pomoci techniky HyperNEAT generuje simulaci primitivního mozku, který chůzi robota ovládá.\par 
 
{\Huge{POŘEŠIT, PŘELOŽIT}} 
%samotna prace 
\title{Vývoj chůze pomocí neuroevolučních algoritmů} 
\author{Jan Bouček} 
\date{} 
\maketitle 
 
\section{Úvod} 
Jeden z cílů dnesšní robotiky je konstruovat roboty, kteří jsou všestranně využitelní díky své mobilitě. Toho lze dosáhnout různými způsoby, pro svou jednoduchost se často používá jízda, která ale neumožňuje spolehlivé zdolávání členitého terénu, nebo pohyb přímo ve strukturách stavěných člověkem, zejména pak zdolávání schodů.\par
Tato práce se inspiruje čtyřnohými kráčivými roboty firmy Boston Dynamics\cite{bostondynamics}, kteří dosahují stabilní chůze exaktními metodami.\cite{bdpaper} Podobné chůze by mělo být možné dosáhnout i pomocí organičtějších metod. Neuroevoluční algoritmy dokáží nabídnout přirozené řešení i na velmi komplexní problémy, či dokonce znovuobjevit řešení, které v přírodě už existuje, proto je možné generovat čtyřnohý pohyb i tímto způsobem.\cite{clunegait}\par
Cílem projektu pak je dosáhnout podobnou metodou stabilní, vzpřímené a repetitivní kupředné chůze na simulaci čtyřnohého robota.
 
\section{Teorie}
Chůzi lze definovat jakožto sérii stavů kráčivého tělesa, kde každý stav lze zjednodušit jako soubor informací popisující jednotlivé pohyblivé prvky robota. V rámci tohoto projektu budeme pracovat s dvojrozměrnou simulací čtyřnohého robota, jehož každá noha je ovládána dvěmi klouby. Stav robota tak lze vystihnout jako osmirozměrný vektor $\vec{s}$. Hledáme pak ovladač robota, který pro každý stav malezne osmirozměrný vektor $\vec{\omega}$, který každému kloubu určuje úhlovou rychlost.
\subsection{Evoluční algoritmy}
Evoluční algoritmy jsou metodou, která dokáže mnohnásobnou selekcí, mutací a kombinací různých řešení postupně vyvíjet optimálnější. K tomu potřebuje funkci, která dokáže každé řešení ohodnotit, tzv. \emph{fitness funkci}. V našem případě hodnotíme maximální vzdálenost robota od počátku směrem do kladných hodnot $x$, a vzhledem k tomu, že cílem je chůze vzpřímená, tak u každého řešení, ve kterém některý článek robota klesne svou výškou těžiště pod experimentálně zjištěnou hodnotu $Y_{min}$, nastavíme výslednou \emph{fitness} ovladače na \emph{0}.\par
Podobně jako v přírodě, evoluční algoritmus si udržuje \emph{populaci} řešení. V každém kroku ohodnotí výše zmíněnou funkcí každého \emph{živočicha}, podle své úspěšnosti jsou pak rozmnoženi (pohlavně, či nepohlavně) a zmutováni. Tento proces je podrobněji popsán v následující sekci.
\subsection{Neuroevoluce}
Neuroevoluční algoritmy jsou specifickou kategorií, která se zabývá trénováním umělých neuronových sítí, anglicky \emph{Artificial Neural Network} (ANN), které jsou abstraktním příblížením dějů v mozku, měly by být tedy schopné sloužit i jako ovladač našeho robota. Jejich struktura je popsána níže.
\subsubsection{Neuronové sítě}
Základní stavební jednotkou neuronové sítě je \emph{neuron}, ten je výpočetní jednotkou, která přijímá data ve formě spojů (z biologického hlediska \emph{dendrity}) s dalšími neurony. Pro každý z nich má nastavenou \emph{váhu} (anglicky \emph{weight}), která slouží jednoduše jako koeficient, který ovlivňuje \uv{důležitost} konkrétního spoje.\par
Výstupem, který pak mohou použít další neurony, je výsledek \emph{aktivační funkce} lineární kombinace hodnot napojených neuronů a příslušných vah\cite{neuron}:
\begin{center}$y_k=\phi(\sum_{j=0}^m w_{kj}*x_j)$\end{center}
\section{Výsledky} 
\section{Implementace?}%sem? 
\section{Závěr} 
\section{Uživatelská dokumentace} 
\section{Diskuze?}
\begin{thebibliography}{4}
\bibitem{bostondynamics}
Introducing Spot. In: \textit{Youtube} [online]. 9.2.2015 [cit. 2017-03-20]. Dostupné z: https://www.youtube.com/watch?v=M8YjvHYbZ9w. Kanál uživatele Boston Dynamics
\bibitem{bdpaper}RAIBERT, Marc, Kevin BLANKESPOOR, Gabriel NELSON a Rob PLAYTER. \textit{BigDog, the Rough-Terrain Quadruped Robot} [online]. [cit. 2017-03-20]. DOI: 10.3182/20080706-5-KR-1001.01833. ISBN 10.3182/20080706-5-KR-1001.01833. Dostupné z: http://linkinghub.elsevier.com/retrieve/pii/S1474667016407020
\bibitem{clunegait}
CLUNE, Jeff, Benjamin E. BECKMANN, Charles OFRIA a Robert T. PENNOCK. Evolving coordinated quadruped gaits with the HyperNEAT generative encoding. In: \textit{2009 IEEE Congress on Evolutionary Computation}. IEEE, 2009, s. 2764-2771. DOI: 10.1109/CEC.2009.4983289. ISBN 978-1-4244-2958-5. Dostupné také z: http://ieeexplore.ieee.org/document/4983289/
\bibitem{neuron}
Artificial neuron. In: \textit{Wikipedia: the free encyclopedia} [online]. San Francisco (CA): Wikimedia Foundation, 2001-2017 [cit. 2017-03-21]. Dostupné z: https://en.wikipedia.org/wiki/Artificial\_neuron
\end{thebibliography}
\end{document}